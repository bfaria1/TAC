\documentclass[12pt]{extarticle}
\usepackage[brazil]{babel}
\usepackage{graphicx, hyperref}
\usepackage{framed}
\usepackage[normalem]{ulem}
\usepackage{amsmath}
\usepackage{amsthm}
\usepackage{amssymb}
\usepackage{amsfonts}
\usepackage{enumitem}
\usepackage{enumerate}
\usepackage[utf8]{inputenc}
\usepackage[top=1 in,bottom=1in, left=1 in, right=1 in]{geometry}
\usepackage{graphicx}
\usepackage[skip=2pt]{caption}
\usepackage{blindtext}
\usepackage[fixlanguage]{babelbib}
\captionsetup{font={small}} 
\graphicspath{{Figuras/}}
\usepackage{appendix}
\newcommand{\asp}[1]{``#1"}
\usepackage[colorinlistoftodos]{todonotes}
\newcommand{\nota}[1]{\todo[color=black!10, bordercolor=white!100, linecolor = black!20, size =\scriptsize]{#1} }
\definecolor{mypink1}{rgb}{0.858, 0.188, 0.478}
\definecolor{mypink2}{RGB}{219, 48, 122}
\definecolor{mypink3}{cmyk}{0, 0.7808, 0.4429, 0.1412}
\definecolor{mygray}{gray}{0.6}
\newcommand{\arb}{um elemento arbitrário }
\newenvironment{resposta}{ \color{mygray}}{}
\newcommand{\true}{\textcolor{red}{\textbf{\textit{V}}}}
\newcommand{\false}{\textcolor{red}{\textbf{\textit{F}}}}
\newcommand{\keys}[1]{\{#1\}}
\newcommand{\subscript}[2]{$#1 _ #2$}
\newcommand{\definition}[3][x]{\{#1|#1 \in \mathbb{#2},#3\}}
\newcommand{\natura}{\mathbb{N}}
\newcommand{\integer}{\mathbb{Z}}
\newcommand{\real}{\mathbb{R}}
\newenvironment{pif}[3][1]{
Tomando $n = #1$ então $#2$ e $#3$

Agora vamos assumir que a expressão é valida para $#1 > n \leq k$ e vamos provar que funciona para $n=k+1$.
}{}
\newcommand{\arranjo}[2]{A_{#1,#2}=\frac{#1!}{(#1-#2)!}}
\newcommand{\arranjoform}[2]{\frac{#1!}{(#1-#2)!}}
\newcommand{\conbination}[2]{C_{#1,#2}=\frac{#1!}{#2!(#1-#2)!}}
\newcommand{\conbinationform}[2]{\frac{#1!}{#2!(#1-#2)!}}
\newcommand{\rconbination}[2]{CR_{#1,#2}=\frac{(#1+#2-1)!}{#2!(#1-1)!}}
\newcommand{\clinear}[1]{\foreach \i in {1,...,#1}{x_\i+}}
\newcommand{\fatorial}[2]{\foreach \i in {#1,...,#2}{\i.}}
\newcommand{\soma}[2]{\sum_{n=#1}^{\infty}#2}
\newcommand{\texto}{\color{mygray}\blindtext\color{black}}
\newcommand{\fim}{\begin{flushright}
   \emph{q.e.d}
\end{flushright}}
%--------------%--------------%----------------------%-----------%--------------

\title{Teoria Axiomática dos Conjuntos}
\author{Beatriz de Faria, 11201810015}
\date{Fevereiro, 2021}

\begin{document}

\maketitle

{\Large\textbf{Lista de Axiomas:}}

\begin{enumerate}[(A)]
    \item  \label{(A)} Existencionalidade: $\forall x \forall y (x = y \Leftrightarrow \forall z (z \in x \Leftrightarrow z \in y))$
    \item \label{(B)} Vazio: $\exists x \forall y (\neg(y \in x))$
    \item \label{(C)} Especificação: $\forall x \exists y \forall z (z \in  y \Leftrightarrow (z \in x \land \varphi(z)))$
    \item \label{(D)} Par: $\forall x \forall y \exists z \forall w (w \in z \Leftrightarrow (w = x \lor w = y))$
    \item \label{(E)} União: $\forall x \exists y \forall z (z \in y \Leftrightarrow \exists w (w \in x \land z \in w))$
    \item \label{(F)} Partes: $\forall x \exists y \forall z (z \in y \Leftrightarrow \forall w (w \in z \Rightarrow w \in x))$
%    \item \label{(G)}
%    \item \label{(H)}
%    \item \label{(i)}
%    \item \label{(J)}
\end{enumerate}

\section{Exercício 2.1.}

Suponha, por absurdo, que existem 2 conjuntos ($x$,$x'$) $x$ $\neq$ $x'$ que satisfazem a propriedade:

\begin{equation}
   \forall y (y \notin Z)
\end{equation}\label{(1)}

Sendo Z, x ou x'.

Como $x$ $\neq$ $x'$, pelo axioma (\ref{(A)}) temos que:

$$\exists a \in x (a \notin x')$$

(Ou vice-versa, mas, podemos supor a proposição acima sem perda de generalidade.)

Porém, como a equação (\ref{(1)}) é válida $\forall y$, ela é válida, em particular para a.
$$\therefore a \in x \land a \notin x$$

Portanto, não existe tal a.

\fim

\section{Exercício 2.26}

\section{Exercício 2.27}

\subsection{i}

\subsection{iii}

\subsection{vi}

\section{Exercício 2.30}

\section{Exercício 2.36}

\section{Exercício 2.40}

\section{Exercício 2.52}

\end{document}
