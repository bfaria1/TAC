\documentclass[12pt]{extarticle}
\usepackage[brazil]{babel}
\usepackage{graphicx, hyperref}
\usepackage{framed}
\usepackage[normalem]{ulem}
\usepackage{amsmath}
\usepackage{amsthm}
\usepackage{amssymb}
\usepackage{amsfonts}
\usepackage{enumitem}
\usepackage{enumerate}
\usepackage[utf8]{inputenc}
\usepackage[top=1 in,bottom=1in, left=1 in, right=1 in]{geometry}
\usepackage{graphicx}
\usepackage[skip=2pt]{caption}
\usepackage{blindtext}
\usepackage[fixlanguage]{babelbib}
\usepackage{chngcntr}
\usepackage{calrsfs}
\captionsetup{font={small}} 
\graphicspath{{Figuras/}}
\usepackage{appendix}
\newcommand{\asp}[1]{``#1"}
\usepackage[colorinlistoftodos]{todonotes}
\newcommand{\nota}[1]{\todo[color=black!10, bordercolor=white!100, linecolor = black!20, size =\scriptsize]{#1} }
\definecolor{mypink1}{rgb}{0.858, 0.188, 0.478}
\definecolor{mypink2}{RGB}{219, 48, 122}
\definecolor{mypink3}{cmyk}{0, 0.7808, 0.4429, 0.1412}
\definecolor{mygray}{gray}{0.6}
\newcommand{\arb}{um elemento arbitrário }
\newenvironment{resposta}{ \color{mygray}}{}
\newcommand{\true}{\textcolor{red}{\textbf{\textit{V}}}}
\newcommand{\false}{\textcolor{red}{\textbf{\textit{F}}}}
\newcommand{\keys}[1]{\{#1\}}
\newcommand{\subscript}[2]{$#1 _ #2$}
\newcommand{\definition}[3][x]{\{#1|#1 \in \mathbb{#2},#3\}}
\newcommand{\natura}{\mathbb{N}}
\newcommand{\integer}{\mathbb{Z}}
\newcommand{\real}{\mathbb{R}}
\newenvironment{pif}[3][1]{
Tomando $n = #1$ então $#2$ e $#3$

Agora vamos assumir que a expressão é valida para $#1 > n \leq k$ e vamos provar que funciona para $n=k+1$.
}{}
\newcommand{\arranjo}[2]{A_{#1,#2}=\frac{#1!}{(#1-#2)!}}
\newcommand{\arranjoform}[2]{\frac{#1!}{(#1-#2)!}}
\newcommand{\conbination}[2]{C_{#1,#2}=\frac{#1!}{#2!(#1-#2)!}}
\newcommand{\conbinationform}[2]{\frac{#1!}{#2!(#1-#2)!}}
\newcommand{\rconbination}[2]{CR_{#1,#2}=\frac{(#1+#2-1)!}{#2!(#1-1)!}}
\newcommand{\clinear}[1]{\foreach \i in {1,...,#1}{x_\i+}}
\newcommand{\fatorial}[2]{\foreach \i in {#1,...,#2}{\i.}}
\newcommand{\soma}[2]{\sum_{n=#1}^{\infty}#2}
\newcommand{\texto}{\color{mygray}\blindtext\color{black}}
\newcommand{\fim}{\begin{flushright}

   \emph{q.e.d}
\end{flushright}}
\counterwithin*{equation}{section}
\counterwithin*{equation}{subsection}
%--------------%--------------%----------------------%-----------%--------------

\title{Teoria Axiomática dos Conjuntos}
\author{Beatriz de Faria, 11201810015}
\date{Fevereiro, 2021}

\begin{document}

\maketitle

\section{Exercício 2.1.}

Suponha, por absurdo, que existem 2 conjuntos ($x$,$x'$) $x$ $\neq$ $x'$ que satisfazem a propriedade:

\begin{equation}
   \forall y (y \notin Z)
\end{equation}

Sendo Z, x ou x'.

Como $x$ $\neq$ $x'$, pelo axioma da existensionalidade temos que:

$$\exists a \in x (a \notin x')$$

(Ou vice-versa, mas, podemos supor a proposição acima sem perda de generalidade.)

Porém, como a afirmação (1) é válida $\forall y$, ela é válida, em particular para a.
$$\therefore a \in x \land a \notin x$$

Portanto, não existe tal a.

\fim

\section{Exercício 2.26}

Para que $a \neq \emptyset \land \bigcap a = \bigcup a$, precisamos encontrar o conjunto a que satisfaça:

\begin{equation}
    \forall z (z \in \bigcup a \Leftrightarrow \exists w (w \in a \land z \in w))
\end{equation}

\begin{equation}
    \forall z (z \in \bigcap a \Leftrightarrow \forall w \in a (z \in w))
\end{equation}

Pelo Axioma da existensionalidade:

$$
\bigcap a = \bigcup a \Leftrightarrow \forall z (z \in \bigcap a \Leftrightarrow z \in \bigcup a)
$$

Em outras palavras, (1) ocorre $\Leftrightarrow$ (2) ocorre:

$$
\exists w (w \in a \land z \in w) \Leftrightarrow \forall w \in a (z \in w)
$$

Note que, se existisse um conjunto $w' \in a$ tal que $\exists s \in w' (s \notin w)$. Logo $s \in \bigcup a$ e $s \notin \bigcap a$, o que contradiz a afirmação acima.

$$
\therefore \forall w \in a (w = w)
$$

Então podemos classificar a como:

$$a = \{w\}$$

Para um conjunto w qualquer.

\section{Exercício 2.27}

\subsection{i} 

$$
    a \subseteq \wp (\bigcup a)
$$

Queremos provar a equação acima, primeiramente, podemos reescreve-la como:

$$\forall x \in a  \Rightarrow x \in \wp (\bigcup a)$$

Para demonstrar essa proposição, primeiramente, fixe um $x \in a$ qualquer. Dizer que $x \in \wp (\bigcup a)$ é equivalente à dizer que $x \subseteq \bigcup a$. Isto é:

$$\forall y \in x (y \in \bigcup a)$$

Tome um $y \in x$ qualquer, como $x \in a$, então $y \subseteq a$. Ou seja, precisamos mostrar que $a \in \bigcup a$. Para tanto, vamos lembrar a definição de $\bigcup a$

$$\forall z (z \in \bigcup a \Leftrightarrow \exists w (w \in a \land z \in w))$$

Em outras palavras, para mostrar que $a \in \bigcup a$, precisamos de um conjunto w tal que $w \in a \land \forall z \in w (z \in a)$. Note que a existência de w é garantida pelo axioma da especificação: 

$$\forall a \exists w \forall z (z \in w \Leftrightarrow (z \in a \land \varphi(z)))$$

No nosso caso:

$$\forall a \exists w \forall z (z \in w \Leftrightarrow (z \in a))$$

\fim

\subsection{iii}

Primeiramente, vamos mostrar que $\bigcap (a \cup b) \subseteq (\bigcap a) \cap (\bigcap b)$. Tome um $x \in \bigcap (a \cup b)$ qualquer. Por definição, sabemos que:

$$
\forall w \in a \cup b (x \in w)
$$

$$
\forall w ( w \in a \lor w \in b (x \in w))
$$

$$
\forall w \in a (x \in w) \land \forall w \in b (x \in w)
$$

$$
x \in \bigcap a \land x \in \bigcap b
$$

$$
\therefore x \in (\bigcap a) \cap (\bigcap b)
$$

Agora, vamos mostrar que $(\bigcap a) \cap (\bigcap b) \subseteq \bigcap (a \cup b)$. Para tanto tome um $x \in (\bigcap a) \cap (\bigcap b)$ arbitrário.

$$
x \in \bigcap a \land x \in \bigcap b
$$

$$
\forall w \in a (x \in w) \land \forall w \in b (x \in w)
$$

$$
\forall w (w \in a \lor w \in b (x \in w))
$$

$$
\forall w \in a \cup b (x \in w)
$$

$$
x \in \bigcap (a \cup b)
$$

\fim 

\subsection{vi}

Primeiramente, vamos provar a ida ($\Rightarrow$)

$$\forall z(z \in \bigcup a \Leftrightarrow \exists x (x \in a \land z \in x)$$

Portanto, podemos reescrever $\bigcup a \subseteq \bigcap b$ como (peço desculpas porque vou escrever mesmo, em linguagem matemática me pareceu um pouco confuso):

Para todo z tal que $z \in x \land x \in w$, $z \in y$ para algum $y \in b$

E isto pode ser reescrito, mais uma vez, como:

$$\forall x \in a \forall y \in b (z \in x \Rightarrow z \in y)$$

Ou seja, $x \subseteq y$

Agora, vamos provar a volta ($\Leftarrow$)

\begin{equation}
   \forall z (z \in \bigcup a \Leftrightarrow \exists \alpha \in a \land z \in \alpha) 
\end{equation}

\begin{equation}
 \forall n (n \in \bigcap b \Leftrightarrow \forall y \in b (n \in y))   
\end{equation}

Tome um $z \in \bigcup a$ qualquer. Por hipótese, nós temos que:

$$\forall x \in a \forall y \in b (x \subseteq y)$$

Como isto é válido $\forall x$, em particular, é válido para o conjunto $\alpha$ empregado em (1): 

$$\forall z(z \in \bigcup a \Leftrightarrow \exists x \in a \land z \in x) $$

Como $x \in a$, $\forall y \in b$ ($z \in y$), devido a nossa hipótese. Em outras palavras:

$$z \in \bigcap b$$

\fim

\section{Exercício 2.30}

Suponha, por absurdo, que existem 2 conjuntos $r$ e $r'$ que satisfazem:

$$\forall z (z \in R \Leftrightarrow \exists x \in a \exists y \in b : z = (x,y))$$

Sendo a e b dois conjuntos arbitrários e $R$ um conjunto tal que:

\begin{equation}
  \forall x \in a \forall y \in b  ((x,y) \in R)  
\end{equation} \label{1}

 O Axioma da existensionalidade nos diz que:

$$ r \neq r' \Leftrightarrow \exists z \in r' ( z \notin r) \lor \exists z \in r ( z \notin r')$$

Suponha, sem perda de generalidade que, $\exists z \in r' (z \notin r)$. Em outras palavras, existe, no mínimo, um $z$ tal que:

$$z = (x,y) \land (x,y) \notin r$$

Porém, pela equação (1) todo par ordenado formado pelos conjuntos $a$ e $b$ $\in R$, como $r$ é um conjunto que satisfaz essa equação então $z \in r \land z \notin r$

\fim

\section{Exercício 2.36}

Queremos provar que existem conjuntos $a$ e $b$ tais que:

$$
\forall z ( z \in p \Rightarrow \exists r \in a \exists s \in b (z = (r,s)))
$$

Por hipótese, temos que, $z \in p \Rightarrow \exists x \exists y (z = (x,y))$, em outras palavras queremos provar que 
\begin{equation}
\forall x (z = (x,y) \Rightarrow \exists r (x \in r \land r \in a))    
\end{equation}



\begin{equation}
\forall y (z = (x,y) \Rightarrow \exists s (y \in s \land s \in a))   
\end{equation}

%O axioma da união nos garante que:

%$$
%\forall a \exists x' \forall z (z \in x' \Leftrightarrow \exists r (r \in a \land z \in r))
%$$


%$$
%\forall b \exists y' \forall z (z \in y' \Leftrightarrow \exists r (r \in b \land z \in r))
%$$



Para tanto, precisamos de conjuntos $r$ e $s$ tais que, $\forall x ( z = (x,y) \Rightarrow x \in r)$ \\ e $\forall y ( z = (x,y) \Rightarrow y \in s)$. Note que, pelo axioma do par, podemos construir os conjuntos $r$ e $s$ como: $r = \{x\}$ e $s = \{y\}$. Isto nos permite reescrever a hipótese como:
$$
z \in p \Rightarrow \exists x \in r \exists y \in s (z = (x,y))
$$

Sejam os conjuntos a e b:

$$
a = \bigcup r
$$

Note que, pela definição de $\bigcup r$ isso satisfaz (1).

$$
b = \bigcup s
$$

Note que, pela definição de $\bigcup s$ isso satisfaz (2).

\fim

\section{Exercício 2.40}

Queremos mostrar que $\bigcup c$ é uma função que denotaremos pela letra R, isto é:

$$\forall x \in dom (\bigcup c) \exists! y (xRy) $$

Note que

\begin{equation}
    x \in dom (\bigcup c) \Leftrightarrow \exists f (dom (f) \in c \land x \in dom (f))
\end{equation}

Temos por hipótese que:

\begin{equation}
   \forall g,h \in c (\forall z \in dom(g) \cap dom(h) (g(z) = h(z))) 
\end{equation}

Como isto vale para quaisquer funções, em particular vale para f (da proposição 1)

Isto nos garante que $\exists y (f(x) = y)$, agora resta mostrar que ele é único. Note que, para que $y$ não seja único, precisamos de um $x$ tal que $x \in dom(g) \land x \in dom(h)$, sendo $dom(g)$ e $dom(h)$ o domínio funções quaisquer em c.  (se x não pertence a g ou h, x não está no domínio de c. Se $x$ pertence há apenas $dom(g)$ ou $dom(h)$, então y é único, pois x é único).

Então, fixe um $x$ tal que $x \in dom(g) \land x \in dom(h)$, note que, pela hipótese (proposição 2), $g(x) = h(x) = y$, portanto y é único.

\fim

\section{Exercício 2.52}

Queremos construir um conjunto b de modo que a função $f$ seja uma bijeção de $a$ em $b$:

$$
\forall x \in a \forall y \in a (x \leq y \Leftrightarrow f(x) \trianglelefteq f(y))
$$

Sendo $\leq$ a ordem parcial de $a$ e $\trianglelefteq$ a ordem parcial de $b$ definida como:

$$
\forall x \in b \forall y \in b (x \trianglelefteq y \Leftrightarrow x \subseteq y)
$$

A partir disto podemos reescrever nossa proposição inicial como:

$$
\forall x \in a \forall y \in a (x \leq y \Leftrightarrow f(x) \subseteq f(y))
$$

Ou seja, precisamos construir um conjunto b, de modo que:

$$
\forall x \in a \forall y \in a (x \leq y \Leftrightarrow \exists z1, z2 \in b (z1 \subseteq z2))
$$

Sendo $z1 = f(x) \land z2 = f(y)$. Seja $f$ a função que leva um elemento $x$ de $a$ no conjunto das partes de x, isto é:
$$
\forall x \in a \exists z \in b (z = \wp (x))
$$

Primeiramente, vamos ver se isto funciona, no caso, $$x \leq y \Leftrightarrow f(x) \subseteq f(y)$$

\subsection{($\Rightarrow$)}

\begin{equation}
    \forall x \in a \forall y \in a (x \leq y \Rightarrow \wp (x) \subseteq \wp (y))
\end{equation}

Pela contrapositiva, podemos dizer que $x > y \Leftrightarrow \wp(x) \nsubseteq \wp(y)$.

Nossa ordem em a nos garante que:

$$
xRy \land yRx \Rightarrow x=y
$$

$$
\therefore x > y \Rightarrow xRy \land \neg (yRx)
$$

(tá, eu tô chutando essa parte eu não consigo provar essa afirmação usando apenas $x \neq y \Rightarrow \neg(xRy \lor yRx)$, na real, eu não sei se eu provei ou se $x>y \Rightarrow yRx \land \neg(xRy)$ ou ainda eu deveria usar que $\wp (y) \subset \wp(x)$, de toda forma, foi o que consegui fazer)

Portanto, podemos dividir isto em 2 condições:

$$
    x > y \Rightarrow xRy \Rightarrow \forall z \in R (\exists s \in x \exists r \in y (z = (s,r) = \{\{s\},\{s,r\}\})
$$


$$
     x > y \Rightarrow \neg(yRx) \Rightarrow \exists z \in R (\exists s \in x \forall r \in y (z \neq (r,s) = \{\{r\},\{r,s\}\})
$$

Note que, $\{r,s\} = \{s,r\}$

$$
\therefore \forall z \exists s \in x \forall r \in y (z = \{\{s\},\{s,r\}\} \neq \{\{r\},\{s,r\}\})
$$

$$
\exists \alpha (\alpha \in \{s\} \land \alpha \notin \{r\})
$$

Note que, $\{s\} \in \wp (x) \land \{r\} \in \wp (y)$. Portanto, $\alpha \subseteq \wp(x) \land \alpha \nsubseteq \wp(y) \Rightarrow p(x) \nsubseteq p(y)$

\subsection{($\Leftarrow$)}

\begin{equation}
    \forall x \in a \forall y \in a (\wp (x) \subseteq \wp (y) \Rightarrow  x \leq y)
\end{equation}

\textbf{1. xRx}

Queremos chegar em:

$$
\wp (x) \subseteq \wp (x) \Rightarrow R \subseteq x^2 = \{\{x\},\{x,x\}\} = \{\{x\},\{x\}\}
$$



Note que, $R \subseteq a^2$ e $x \in a$, portanto, $\forall z \in R \exists x \in a  (z = \{\{x\},\{x\}\})$

\textbf{2. xRy $\land$ yRx $\Rightarrow x = y$}

$$
\forall z \in R (\exists s \in x \exists r \in y (z = \{\{s\},\{s,r\}\})
$$

$$
\forall z \in R (\exists s \in x \exists r \in y (z = \{\{r\},\{r,s\}\})
$$

$$
\therefore
$$

$$
\forall z \in R (\exists s \in x \exists r \in y(z = \{\{r\},\{r,s\}\} \land z = \{\{s\},\{s,r\}\} ))
$$

Ok, aqui eu desisto, desculpa :(

\subsection{f(x) é uma bijeção}

Agora, vamos provar que essa função é uma bijeção, isto é:

$$
\forall z \in b \exists ! x \in a (z = \wp (x))
$$

Sejam $x,y \in a$ e $\wp (x) = \wp (y)$, temos que:

$$
z \in \wp (x) \Leftrightarrow z \subseteq x 
$$

como $\wp(y) = \wp(x)$

$$
z \in \wp (y) \Leftrightarrow z \subseteq x
$$

pela definição de conjunto das partes:

$$
 z \subseteq y \Leftrightarrow z \in \wp (y) \Leftrightarrow z \subseteq x 
$$

Ou seja, $z \in y \Leftrightarrow z \in x$. Pelo axioma da existensionalidade $y = x$

\end{document}

