\documentclass[12pt]{extarticle}
\usepackage[brazil]{babel}
\usepackage{graphicx, hyperref}
\usepackage{framed}
\usepackage[normalem]{ulem}
\usepackage{amsmath}
\usepackage{amsthm}
\usepackage{amssymb}
\usepackage{amsfonts}
\usepackage{enumitem}
\usepackage{enumerate}
\usepackage[utf8]{inputenc}
\usepackage[top=1 in,bottom=1in, left=1 in, right=1 in]{geometry}
\usepackage{graphicx}
\usepackage[skip=2pt]{caption}
\usepackage{blindtext}
\usepackage[fixlanguage]{babelbib}
\usepackage{chngcntr}
\usepackage{calrsfs}
\captionsetup{font={small}} 
\graphicspath{{Figuras/}}
\usepackage{appendix}
\newcommand{\asp}[1]{``#1"}
\usepackage[colorinlistoftodos]{todonotes}
\newcommand{\nota}[1]{\todo[color=black!10, bordercolor=white!100, linecolor = black!20, size =\scriptsize]{#1} }
\definecolor{mypink1}{rgb}{0.858, 0.188, 0.478}
\definecolor{mypink2}{RGB}{219, 48, 122}
\definecolor{mypink3}{cmyk}{0, 0.7808, 0.4429, 0.1412}
\definecolor{mygray}{gray}{0.6}
\newcommand{\arb}{um elemento arbitrário }
\newenvironment{resposta}{ \color{mygray}}{}
\newcommand{\true}{\textcolor{red}{\textbf{\textit{V}}}}
\newcommand{\false}{\textcolor{red}{\textbf{\textit{F}}}}
\newcommand{\keys}[1]{\{#1\}}
\newcommand{\subscript}[2]{$#1 _ #2$}
\newcommand{\definition}[3][x]{\{#1|#1 \in \mathbb{#2},#3\}}
\newcommand{\natura}{\mathbb{N}}
\newcommand{\integer}{\mathbb{Z}}
\newcommand{\real}{\mathbb{R}}
\newenvironment{pif}[3][1]{
Tomando $n = #1$ então $#2$ e $#3$

Agora vamos assumir que a expressão é valida para $#1 > n \leq k$ e vamos provar que funciona para $n=k+1$.
}{}
\newcommand{\arranjo}[2]{A_{#1,#2}=\frac{#1!}{(#1-#2)!}}
\newcommand{\arranjoform}[2]{\frac{#1!}{(#1-#2)!}}
\newcommand{\conbination}[2]{C_{#1,#2}=\frac{#1!}{#2!(#1-#2)!}}
\newcommand{\conbinationform}[2]{\frac{#1!}{#2!(#1-#2)!}}
\newcommand{\rconbination}[2]{CR_{#1,#2}=\frac{(#1+#2-1)!}{#2!(#1-1)!}}
\newcommand{\clinear}[1]{\foreach \i in {1,...,#1}{x_\i+}}
\newcommand{\fatorial}[2]{\foreach \i in {#1,...,#2}{\i.}}
\newcommand{\soma}[2]{\sum_{n=#1}^{\infty}#2}
\newcommand{\texto}{\color{mygray}\blindtext\color{black}}
\newcommand{\fim}{\begin{flushright}

   \emph{q.e.d}
\end{flushright}}
\counterwithin*{equation}{section}
\counterwithin*{equation}{subsection}
%--------------%--------------%----------------------%-----------%--------------

\title{Teoria Axiomática dos Conjuntos}
\author{Beatriz de Faria, 11201810015}
\date{Abril, 2021}

\begin{document}

\maketitle

\section{Exercício 5.16.}

\subsection{$\delta$ é um ordinal}

Para mostrarmos que $\delta$ é um ordinal, vamos dividir em 3 partes:

\subsubsection{$\delta$ é transitivo}

Tome um $x \in \delta$ qualquer. Por hipótese temos que:

$$
\forall x (x \in \delta \Leftrightarrow \forall \alpha \in B (x \in \alpha))
$$

Em particular, temos que $x \in \alpha$. Note que, como $B$ é um conjunto de ordinais, $\alpha$ é um ordinal, portanto:

$$
\forall x \in \alpha (\wp(x) \in \alpha)
$$

Como isto é válido $\forall \alpha$ então, $\wp(x) \in \delta$.

\subsubsection{$\forall x \in \delta (x \notin x)$}

Tome um $x \in \delta$ qualquer, pelo mesmo raciocínio do item anterior, sabemos que $x \in \alpha$ e $\alpha$ é um ordinal. Como $\alpha$ é um ordinal:

$$
\forall y \in \alpha (y \notin y)
$$

Como isto é válido $\forall y$, em particular, vale para $x$. Portanto $x \notin x$

\subsubsection{$\forall x,y \in \delta (x \leq y \Leftrightarrow (x \in y \lor y = x)$}

Tome $x,y \in \delta$ quaisquer. Sabemos que $x,y \in \alpha (\forall \alpha \in B)$, como $\alpha$ é um ordinal:

$$
\forall z,w \in \alpha (z \leq w \Leftrightarrow (z \in w \lor z = w))
$$

Como isto é válido para quaisquer $z,w$ podemos dizer, sem perda de generalidade, que isto vale para $z = x$ e $w = y$. Temos que:

$$
\forall x,y \in \alpha (x \leq y \Leftrightarrow (x \in y \lor x = y))
$$

Como queríamos.

\subsection{$\delta \in B$}

Sabemos (na real ainda não sabemos, mas decorre da parte 1.3 deste exercício) que $\delta \in \alpha \lor \delta = \alpha$ para todo $\alpha \in B$. 

\textbf{Caso 1. $\delta \in \alpha$}

Como $\alpha \in B$ e $\delta \in \alpha$ logo,

$$
\delta \subseteq B
$$

Note que, se ocorre $\delta = B$, então $\alpha \in \delta$ e como $\delta$ é um ordinal, $\alpha \subseteq \beta$, ou seja $\delta = \alpha$, o que contradiz nossa hipótese de que $\delta \in \alpha \because \delta \notin \delta$. Portanto, podemos dizer que $\delta \subsetneq B$ e, pela proposição 5.9:

$$
\delta \subsetneq B \Rightarrow \delta \in B
$$

\textbf{Caso 2. $\delta = \alpha$}

Como $\alpha \in B \Rightarrow \delta \in B$ 

\subsection{$\forall \alpha \in B (\delta \leq \alpha)$}

Suponha por absurdo que isto não ocorre, isto é: $\alpha \in \delta$ (proposição 5.12), para algum $\alpha \in B$. Note que,

$$
\forall x (x \in \delta \Leftrightarrow \forall A \in B (x \in A))
$$

Como isto vale $\forall A$, em particular, vale para $\alpha$, portanto:

$$
\forall x \in \delta \Rightarrow x \in \alpha
$$

Como isto vale $\forall x \in \delta$, em particular, vale para $x = \alpha$. Teremos que:

$$
\alpha \in \alpha
$$

$$
\therefore \exists x \in \delta (x \in x)
$$

O que contradiz o fato de que $\delta$ é um ordinal

\fim

\section{Exercício 5.28.}

\subsection{Prove que, se $y \in a$ e $y'\in a$ são sucessores de $x \in a$, então $y = y'$.}

Suponha, por absurdo, que $y \neq y'$. Suponha também, sem perca de generalidade, que:


$$
y \triangleleft y'
$$

Como $x \triangleleft y$, então:

$$
x \triangleleft y \triangleleft y'
$$

O que é um absurdo, pois $y \in a$ e $y'$ é um sucessor de x, então, $\nexists z \in a : x \triangleleft z \triangleleft y'$.

\fim

\subsection{Prove que $(a, \trianglelefteq)$ e $(\omega, \leq)$ são ordens isomorfas}

Queremos mostrar que existe uma função bijetora satisfazendo:

$$
\forall x \in a \forall y \in a (x \trianglelefteq y \Leftrightarrow f(x) \leq f(a))
$$

Construa a função $f: a \rightarrow \omega$ da seguinte maneira:

1. $f(x) = \emptyset$ se $x =$ min$_{\trianglelefteq}(a)$

A existência de um elemento mínimo é garantida pois $\trianglelefteq$ é uma boa ordem sobre a.

2. $f(S(y)) = s(i_y)$

Onde o sucessor de um elemento $y$ de $a$ denotado por $ := S(y)$ (se $x \neq$ min$_{\trianglelefteq}(a)$, então $\exists y : x = S(y)$) e $i_y$ é o y-ésimo elemento de $\omega$, de modo que:
\\
Se $y = $ min$_{\trianglelefteq}(a)$, $s(i) = s(\emptyset)$ \\
Se $y =$ S(min$_{\trianglelefteq}(a)$), $s(i) = s(s(\emptyset))$\\
Se $y =$ S(S(min$_{\trianglelefteq}(a)$)), $s(i) = s(s(s(\emptyset)))$\\
E assim por diante \textit{(espero que tenha ficado claro, acho que é a primeira vez que estou definindo uma função por recursão)}. Vamos provar que essa função é bijetora.

\textbf{1. Ela é injetora}

Sejam $s(i_m), s(i_n) \in \omega $ tais que $s(i_m) = s(i_n)$.

$$
\therefore i_m = i_n \Rightarrow m = n \Rightarrow S(m) = S(n)
$$



\textbf{2. Ela é sobrejetora}

Decorre do fato de que a não possui elemento máximo na ordem $\trianglelefteq$, portanto, $\forall s(i) \in \omega (\exists n \in a: f(S(n) = s(i))$, além disso $\emptyset =$ min$_{\trianglelefteq}(a)$. Então, $\forall m \in \omega (\exists n \in a : m = f(n))$

\fim

\subsection{Mostre, por meio de contraexemplos, que o resultado do item 2.2 seria falso se qualquer uma das três condições listadas fosse omitida.}

\textbf{1. a possui elemento máximo na ordem $\trianglelefteq$}

Seja $(a, \trianglelefteq)$ um segmento inicial próprio de $\omega$ (note que este conjunto respeita as demais condições impostas sobre a), temos pela proposição 4.15 que o resultado anterior é falso. 

\textbf{2. $\trianglelefteq$ não é uma boa-ordem sobre a}

Se $\forall x \in a (x = s(\alpha))$ para algum $\alpha \in a$, então $a$ não é indutivo pois $\emptyset \notin a$. Se ocorre:

$$
\exists x \in a : f(x) = \emptyset
$$

Note que $x = s(\alpha)$, portanto, $f(\alpha) = n$ e $n \in \emptyset$ (pela ordem de $\omega$), o que é um absurdo


\textbf{3. $\neg$ (para todo $y \in a$, se não existe $x \in a$ tal que y é o sucessor de x, então y = min$_{\trianglelefteq}(a)$)}

Seja $a = \omega \cup \{\omega\}$

Se $a$ e $\omega$ fossem ordens isomorfas, então existe uma função bijetora de modo que:

$$
\exists x \in \omega : f(x) = \omega
$$

Como $\omega$ é indutivo $s(x) \in \omega$, então, precisamos de $c \in a$ de modo que $\omega \triangleleft c$ (para respeitar que as ordens sejam isomorfas), e dado que $a = \omega \cup \{\omega\}$ e $c \neq \omega$, portanto $c \in \omega \Rightarrow \omega \subsetneq \omega$, o que é um absurdo.

\section{Exercício 5.44.}

\subsection{$s(\alpha) \in \beta$}

Vamos provar que se $s(\alpha) \notin \beta \Rightarrow \alpha \notin \beta$

Sabemos que $s(\alpha) = \alpha \cup \{\alpha\}$. Como $s(\alpha) \notin \beta$, $s(\alpha) \nsubseteq \beta$ (note que $\beta \neq s(\alpha)$ pois $\beta$ é um ordinal limite), então $\exists x : x \in s(\alpha) \land x \notin \beta$, o que é o mesmo que dizer que:

$$
\alpha \notin \beta \lor \{\alpha\} \notin \beta
$$


\textbf{Caso 1. $\alpha \notin \beta$}

É imediado

\textbf{Caso 2. $\{\alpha\} \notin \beta$}

Se $\{\alpha\} \notin \beta$ então, dado que beta é um ordinal, $\alpha \notin \beta \because \forall y \in \beta (y \subseteq \beta)$ 


\subsection{$\Rightarrow$}

Suponha, por absurdo, que para um ordinal limite $\beta$, $\exists m \in \beta : \forall x \in \beta (x \leq m)$. Como isto vale $\forall x$, em particular, vale para $x = m$, porém, como $m \in \beta \Rightarrow s(m) \in \beta$ e $m < s(m)$, o que é um absurdo. 

\subsection{$\Leftarrow$}

Seja $a = s(\alpha)$ para algum $\alpha$. Queremos mostrar que $\forall x \in a \exists m \in a (x \leq m)$. Seja $m = \{\alpha\}$, como $a = s(\alpha) = \alpha \cup \{\alpha\}$. 

\textbf{Caso 1. x = $\alpha$}

$$x \in \{\alpha\} \therefore x < m \Rightarrow x \leq m$$

\textbf{Caso 2. x = $\{\alpha\}$}

$$
x = m \Rightarrow x \leq m
$$

\fim

\section{Exercício 5.47.}

Seja $\alpha$ um conjunto de ordinais que não possui elemento máximo. Queremos mostrar que $\bigcup \alpha$ é um ordinal limite.

$$
x \in \bigcup \alpha \Leftrightarrow \exists z (z \in \alpha \land x \in z)
$$

Note que, $\nexists m \in \alpha \forall z \in \alpha   : z \leq m$, portanto $m \notin \bigcup \alpha$, então, $\bigcup \alpha$ não possui elemento máximo e, vimos no exercício 5.44 que isso ocorre somente se $\bigcup \alpha$ é um ordinal limite.  

\fim

\section{Exercício 6.17.}

Queremos mostrar que:

$$
\forall n \in \omega (n \precsim a  \backslash t)
$$

Seja o conjunto:

$$
\alpha = \{n \in \omega : n \precsim a  \backslash t\}
$$

Note que, $\forall n \in \alpha \Rightarrow n \in \omega \therefore \alpha \subseteq \omega$, ou seja, se demonstrarmos que $\alpha$ é indutivo, pelo produto da indução finita, teremos o resultado esperado.

\textbf{Parte 1:}

$$
\emptyset \in \alpha
$$

Queremos que exista uma $f:\emptyset \rightarrow a \backslash t$ de modo que $\forall x,y \in \emptyset (x \neq y \Rightarrow f(x) \neq f(y))$. O que vale por vacuidade.

\textbf{Parte 2:}

$$
\forall n \in \alpha (s(n) \in \alpha)
$$

Fixe um $n \in \alpha$ qualquer, sabemos que existe uma função $f: n \rightarrow a \backslash t$ de modo que $\forall x, y \in n (x \neq y \Rightarrow f(x) \neq f(y))$. Queremos mostrar que existe uma função $g: s(n) \rightarrow a \backslash t$ de modo que $\forall x, y \in s(n) (x \neq y \Rightarrow g(x) \neq g(y))$

Sabemos que $s(n) = n \cup \{n\}$, se $x \in n$, sabemos pela nossa hipótese da indução que isto vale, ou seja $g(x) = f(x)$ se $x \in n$. Queremos demonstrar que $\exists \beta \in a \backslash t : \beta \notin f[n]$, e assim, poderemos construir a função de modo que se $x = n \Rightarrow g(x) = \beta$.

Por hipótese: $\exists r \in \omega : t \approx r$. Além disso, $a$ é infinito, então $\omega \precsim a$. Portanto, $\omega \backslash r \precsim a \backslash t$. Então, o que queremos provar é, na verdade:

$$
\exists \beta' \in \omega \backslash r : \beta' \notin f[n]
$$

Suponha, por absurdo que:

$$
\forall \beta' \in \omega \backslash r ( \exists z \in n : f(z) = \beta')
$$

Note que, para que isto ocorra, precisamos que:

$$
\forall \beta' \in \omega (\exists z \in n + r : f(z) = \beta') 
$$

Como $n,r \in \omega$, então $n+r \in \omega$, então, nosso resultado contradiz a proposição 6.12. Garantindo a existência de $\beta$

\fim

\section{Exercício 6.18.}

\textit{Eu tentei fazer pela sugestão que você deu e não deu certo, então eu meio que chutei, com certeza está errado}

Queremos demonstrar que $f[t]$ é finito. Para tanto, suponha por absurdo que:

$$
f[t] \approx \omega
$$

Temos por hipótese que:

$$
t \approx n \in \omega
$$

Sabemos que:

$$
y \in f[t] \Leftrightarrow \exists x \in t : f(x) = y
$$

Sejam $g:f[t] \rightarrow \omega$ e $h: t \rightarrow n$ funções bijetoras. Temos:

$$
y \in g(f[t]) \Leftrightarrow \exists x \in g(t): g(f(x)) = y
$$

Note que $y \in g(f[t]) \Leftrightarrow y \in \omega$ (pois a função é bijetora)

$$
\therefore y \in \omega \Leftrightarrow \exists h(x) \in h(g(t)) : h(g(f(x))) = h(y)
$$

$$
h(y) \in n \Rightarrow \omega \subseteq n
$$

O que é um absurdo.

\fim

\textit{Erradasso, eu sei, só não quis deixar em branco mesmo :S}
\end{document}
