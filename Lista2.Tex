\documentclass[12pt]{extarticle}
\usepackage[brazil]{babel}
\usepackage{graphicx, hyperref}
\usepackage{framed}
\usepackage[normalem]{ulem}
\usepackage{amsmath}
\usepackage{amsthm}
\usepackage{amssymb}
\usepackage{amsfonts}
\usepackage{enumitem}
\usepackage{enumerate}
\usepackage[utf8]{inputenc}
\usepackage[top=1 in,bottom=1in, left=1 in, right=1 in]{geometry}
\usepackage{graphicx}
\usepackage[skip=2pt]{caption}
\usepackage{blindtext}
\usepackage[fixlanguage]{babelbib}
\usepackage{chngcntr}
\usepackage{calrsfs}
\captionsetup{font={small}} 
\graphicspath{{Figuras/}}
\usepackage{appendix}
\newcommand{\asp}[1]{``#1"}
\usepackage[colorinlistoftodos]{todonotes}
\newcommand{\nota}[1]{\todo[color=black!10, bordercolor=white!100, linecolor = black!20, size =\scriptsize]{#1} }
\definecolor{mypink1}{rgb}{0.858, 0.188, 0.478}
\definecolor{mypink2}{RGB}{219, 48, 122}
\definecolor{mypink3}{cmyk}{0, 0.7808, 0.4429, 0.1412}
\definecolor{mygray}{gray}{0.6}
\newcommand{\arb}{um elemento arbitrário }
\newenvironment{resposta}{ \color{mygray}}{}
\newcommand{\true}{\textcolor{red}{\textbf{\textit{V}}}}
\newcommand{\false}{\textcolor{red}{\textbf{\textit{F}}}}
\newcommand{\keys}[1]{\{#1\}}
\newcommand{\subscript}[2]{$#1 _ #2$}
\newcommand{\definition}[3][x]{\{#1|#1 \in \mathbb{#2},#3\}}
\newcommand{\natura}{\mathbb{N}}
\newcommand{\integer}{\mathbb{Z}}
\newcommand{\real}{\mathbb{R}}
\newenvironment{pif}[3][1]{
Tomando $n = #1$ então $#2$ e $#3$

Agora vamos assumir que a expressão é valida para $#1 > n \leq k$ e vamos provar que funciona para $n=k+1$.
}{}
\newcommand{\arranjo}[2]{A_{#1,#2}=\frac{#1!}{(#1-#2)!}}
\newcommand{\arranjoform}[2]{\frac{#1!}{(#1-#2)!}}
\newcommand{\conbination}[2]{C_{#1,#2}=\frac{#1!}{#2!(#1-#2)!}}
\newcommand{\conbinationform}[2]{\frac{#1!}{#2!(#1-#2)!}}
\newcommand{\rconbination}[2]{CR_{#1,#2}=\frac{(#1+#2-1)!}{#2!(#1-1)!}}
\newcommand{\clinear}[1]{\foreach \i in {1,...,#1}{x_\i+}}
\newcommand{\fatorial}[2]{\foreach \i in {#1,...,#2}{\i.}}
\newcommand{\soma}[2]{\sum_{n=#1}^{\infty}#2}
\newcommand{\texto}{\color{mygray}\blindtext\color{black}}
\newcommand{\fim}{\begin{flushright}

   \emph{q.e.d}
\end{flushright}}
\counterwithin*{equation}{section}
\counterwithin*{equation}{subsection}
%--------------%--------------%----------------------%-----------%--------------

\title{Teoria Axiomática dos Conjuntos}
\author{Beatriz de Faria, 11201810015}
\date{Março, 2021}

\begin{document}

\maketitle

\section{Exercício 3.3.}

O conjunto $\bigcap A$ é definido por:

$$
x \in \bigcap A \Leftrightarrow \forall a \in A (x \in a)
$$

Queremos provar que:\\

\textbf{(i) $\emptyset \in x$}

De fato, como $ a \in A \Rightarrow$ a é indutivo. Então, pela definição de conjunto indutivo, $\forall a \in A (\emptyset \in a)$,isto é, $\emptyset \in x$. \\

\textbf{(ii) $z \in x \Rightarrow s(z) \in x$}

Fixe um $z \in x$ qualquer, como $z \in x \Rightarrow \forall a \in A (z \in a).$ Só que, $ a \in A \Rightarrow$ a é indutivo. Ou seja, $\forall a \in A (s(z) \in a)$. Portanto $s(z) \in x$.

\fim

\section{Exercício 3.14.}

Por definição:

$$
m \leq n \Leftrightarrow (m \in n \lor m = n)
$$

Disso, temos que:

$$
m < n \Leftrightarrow (m \leq n \land m \neq n)
$$

é equivalente a dizer:

$$
m < n \Leftrightarrow ((m \in n \lor m = n) \land m \neq n)
$$

Note que não pode ocorrer m = n. Portanto, $m \in n$

\fim

\section{Exercício 3.17.}

Fixe $n \in \omega$ qualquer

$(\Rightarrow)$ 

Como $\omega$ é um conjunto indutivo, $\forall x \in \omega, s(x) \in \omega$. Suponha, por absurdo, que $\nexists k : s(k) = n$, além disso, $\forall a$ tal que a é um conjunto indutivo $\omega \subseteq a$. Seja a o conjunto construído da seguinte forma: $\forall y (y \in a \Leftrightarrow y = \emptyset \lor$ y é o sucessor de algum elemento de a). Note que existe um conjunto indutivo que satisfaz as hipóteses colocadas. 

Como $\omega \subseteq a$, então $\forall y \in \omega (y \in a)$. Então, como n não pode ser o sucessor de um elemento k e $\forall y \in \omega (y = \emptyset \lor y = s(k))$ para algum k em $\omega$. Disto temos que $n = \emptyset$ o que contradiz nossa hipótese.

$(\Leftarrow)$ 

Essencialmente a mesma coisa feita anteriormente só que sem a parte do absurdo.

\fim

\section{Exercício 3.18.}

\textbf{(a)}

Fixe um $n \in \omega$ qualquer.

Como $\omega$ é indutivo, $s(n) \in \omega$. Pela nossa definição de sucessor:

$$
s(n) = n \cup \{n\} = \{n ,\{n\}\}
$$

O axioma das partes nos dá que:

$$
\therefore n \in s(n) \land s(n) \in \omega \Rightarrow n \subseteq \omega
$$
\fim
\\
\textbf{(b)}

Seja $n \in \omega : n \neq \emptyset$. Queremos provar que:

$$
n = \{m \in \omega : m < n \} = \{m \in \omega: m \in n\}
$$

Caso 1. $n = \emptyset$

$$
\therefore \nexists m \in n
$$

Nossa proposição é valida por vacuidade \textit{(é? Não sei se provei ou não, mas acho que é isso)}

Caso 2. $n \neq \emptyset$

Pelo exercício 3.17 sabemos que:

$$
\exists k \in \omega : n = s(k) = \{0,1,2,...,k\}
$$

Temos que mostrar que o conjunto $\{0,1,2,...,k\} = \{m \in \omega : m \in n\}$. Para tanto, fixe um $x \in \{0,1,2,...,k\}$ qualquer. Note que $x \in n$ pois $s(k) = n$, devemos mostrar, portanto, que $x \in \omega$. Como $x \in s(k) \land s(k) \in \omega$, então, pela transitividade de $\omega$, $x \in \omega$


Fixe um $y \in \{m \in \omega : m \in n\}$ qualquer, $y \in \omega \land y \in n$ como $y \in n \Rightarrow y \in s(k) \because n = s(k)$
\fim
\\
\textbf{(c)}

($\Rightarrow$)

Temos por hipótese que: $m \in s(n)$ (Lema 3.10). Portanto:

$$
m \in n \cup \{n\}
$$

Caso 1. $m \in \{n\}$

Neste caso, por definição, $m \subseteq n$ 

Caso 2. $m \in n$

Temos, então, pelo lema 3.12 que:

$$
s(m) \leq n
$$

Logo,

$$
s(m) \in n \lor s(m) = n
$$

Caso 2.1. $s(m) \in n$

$$
m \in s(m) \land s(m) \in n
$$

$$
\therefore m \subseteq n
$$

Caso 2.2. $s(m) = n$

$$
n = m \cup \{m\} 
$$

$$
m \in \{m\} \land \{m\} \in n
$$

$$
\therefore m \subseteq n 
$$

($\Leftarrow$)

Temos, por hipótese que $m \subseteq n$, queremos mostrar que $m \in n \lor m = n$. Suponha, por absurdo, que $m \notin n \land m \neq n$. 

\begin{equation}
\exists x : m \in x \land x \subseteq m     
\end{equation}



Ou seja:

$$
\forall y ( y \subseteq a \Leftrightarrow \forall b \in y (b \in a))
$$

Isso quer dizer que podemos reescrever (1) como:

$$
m \in m
$$

Mas isto é um absurdo, pois $m \in \omega$ e omega é um ordinal.

\textit{Não sei se posso usar essa informação aqui D: (em teoria no 3.18 a gente ainda não viu ordinais), mas não consegui provar a partir do item (b)}

\fim





\section{Exercício 4.17.}

\textit{Nem se dê ao trabalho de corrigir este e o próximo exercício, eles estão errados, ou, no mínimo, incompletos}

Queremos provar que o conjunto: $f\upharpoonright_s = \{ (x,y) \in f : x \in s\}$ é um isomorfismo de ordens entre $(s, \trianglelefteq)$ e $(f[s], \preceq)$. Ou seja:

$$
\forall x \in s \forall y \in f[s] (x \trianglelefteq y \Leftrightarrow f\upharpoonright_s(x) \preceq f\upharpoonright_s(y))
$$

Sendo que:

$$
y \in f[s] \Leftrightarrow \exists x \in s (f(x) = y)
$$

Em outras palavras:

$$
\forall x \in s (x \trianglelefteq f(x) \Leftrightarrow f\upharpoonright_s(x) \preceq f\upharpoonright_s(f(x)))
$$




\section{Exercício 4.21.}

\textbf{(a)} Para mostrar que $(S, \preceq)$ é uma ordem total temos que mostrar que:

$$
\forall x \in s \forall y \in s (x \preceq y \lor y \preceq x)
$$

Suponha por absurdo que não ocorre nenhuma das coisas acima, por hipótese temos que:

$$
\forall x, y \in S (x \preceq y \Leftrightarrow x=y \lor( x \neq y \land f(\triangle(x,y))=0))
$$

Portanto, existem $x,y \in S$ tais que:

$$
x \neq y \land (x = y \lor x(\triangle(x,y) \neq 0)) 
$$

$$
x \neq y \land x(\triangle(x,y)) \neq 0
$$

Como $x \in \{0,1\}$

$$
x \neq y \land  x(\triangle(x,y)) = 1
$$

$$
\triangle (x,y) = min \{n \in \omega : x(n) \neq y(n)\}
$$

Note que $x(\triangle(x,y)) = x(min\{n \in \omega: x(n) \neq y(n)\}) = 1$




\section{Exercício 5.5.}

(A ordem das demonstrações nesse exercício vai ficar meio esquisita, mas, vamos lá

\textbf{(i) $\Rightarrow$ (iii)}

Nossa hipótese é que:

$$
(x \leq y \land x \neq y)
$$

Pelo modo como definimos $x \leq y$:

$$
((x \in y \lor x = y) \land x \neq y)
$$

$$
((x \in y \land x \neq y) \lor (x = y \land x \neq y) 
$$

A segunda condição no nosso \textbf{\asp{ou}} não pode ocorrer, portanto, ocorre que:

$$
(x \in y \land x \neq y)
$$
 
 Em particular:
 
$$
x \in y
$$
\\
\textbf{(i) $\Rightarrow$ (ii)}

Queremos provar que:

$$
\neg (y \in x \lor y = x)
$$

que é a mesma coisa que:

$$
(y \notin x \land y \neq x)
$$

Primeiramente, vamos provar que $y \neq x$. Por hipótese, temos: 

$$
x \leq y \land x \neq y
$$

Portanto, $y \neq x$ decorre diretamente da nossa hipótese.

Agora, vamos provar que $y \notin x$. Já sabemos que $(i) \Rightarrow (iii)$, então temos que $x \in y$. Suponha por absurdo, que $y \in x$. Como $\alpha$ é um ordinal e $x \in \alpha$, em particular $x$ é um ordinal

$$
\therefore y \subseteq x 
$$

$$
x \in y \land y \subseteq x \Rightarrow x \in x
$$

O que é um absurdo pois $\alpha$ é um ordinal.
\\
\textbf{(iii) $\Rightarrow$ (i)}
provamos no item anterior que $x \in y \Rightarrow y \notin x$, devemos provar, portanto, que $x \neq y$, suponha por absurdo que $x = y$, teremos que $x \in x$, o que é um absurdo, pois $\alpha$ é um ordinal.
\\
\textbf{(ii) $\Rightarrow$ (iii)}
Temos, por hipótese que:

$$
x \neq y \land y \notin x
$$

Suponha, por absurdo que $x \notin y$

Nesse caso $\neg(xR y) \land \neg (y R x)$. Ou seja, não há relação de ordem em $\alpha$, que é um ordinal, então é um absurdo.
\\
\textbf{(ii) $\Rightarrow$ (i)}

Como (ii) $\Rightarrow$ (iii) e (iii) $\Rightarrow$ (i), então, (ii) $\Rightarrow$ (i)
\\
\textbf{(iii) $\Rightarrow$ (ii)}

Como (iii) $\Rightarrow$ (i) e (i) $\Rightarrow$ (ii), então, (iii) $\Rightarrow$ (ii)

\fim

\section{Exercício 5.13.}

\textit{Eu vou separar as bolinhas em números pra organizar melhor}

\textbf{1. $\alpha = \beta \lor \alpha < \beta \lor \beta < \alpha$}

Primeiramente vamos provar que uma dessas coisas ocorre (na real, não vamos, porque já o fizemos).

A demonstração é a própria proposição 5.12.

Agora, vamos provar que apenas uma dessas coisas ocorre. 

\textbf{1.1. Suponha, por absurdo, que $\alpha = \beta \land \alpha < \beta$}

Como $\alpha = \beta$, $\alpha < \alpha \Rightarrow \alpha \in \alpha$, que é um absurdo, pois $\alpha \in \beta$ que é um ordinal.

\textbf{1.2 Suponha, por absurdo, que $\alpha = \beta  \land \beta < \alpha$}

(Análogo ao anterior)

\textbf{1.3 Suponha, por absurdo, que $ \alpha < \beta \land \beta < \alpha$}

Portanto $\alpha \in \beta$, como $\beta$ é um ordinal $\alpha \subseteq \beta$

$$
\therefore \beta \in \alpha \land \alpha \subseteq \beta
$$

$$
\beta \in \beta
$$

O que é um absurdo, pois $\beta \in \alpha$ e $\alpha$ é um ordinal.

\textbf{2. $\alpha < \beta \Leftrightarrow (\alpha \leq \beta \land \alpha \neq \beta)$}

$(\Rightarrow)$

Por hipótese:

$$
\alpha \in \beta
$$

Por si só, isto já satisfaz que:

$$
(\alpha \in \beta \lor \alpha = \beta) \land (\alpha \neq \beta)
$$

pois isto é o mesmo que dizer que:

$$
\alpha \in \beta
$$

($\Leftarrow$)

Temos que:

$$
(\alpha \in \beta \lor \alpha = \beta) \land (\alpha \neq \beta)
$$

Que é o mesmo que dizer:

$$
\alpha \in \beta
$$

Pois $\alpha$ não pode ser simultaneamente igual e diferente de $\beta$.

\textbf{3. $\alpha \leq \beta \land \beta \leq \gamma \Rightarrow \alpha \leq \gamma$}

Temos que:

$$
\alpha \subseteq \beta \land \beta \subseteq \gamma
$$

$$
\forall x \in \alpha (x \in \beta)
$$

$$
\forall y \in \beta (y \in \gamma)
$$

Como isto é válido para todo y em $\beta$, em particular, é valido para os $x \in a$. Portanto

$$
\forall x \in a (x \in \gamma)
$$

$$
\therefore a \subseteq \gamma
$$

$$
\alpha \leq \gamma
$$

\textbf{4. $\alpha < \beta \land \beta < \gamma \Rightarrow \alpha < \gamma$}

Temos que:

$$
\alpha \in \beta \land \beta \in \gamma
$$

Como $\gamma$ é um ordinal:

$$
\alpha \in \beta \land \beta \subseteq \gamma
$$

$$
\forall x \in \beta (x \in \gamma)
$$

Como isso é valido para todo x, em particular é válido para $\alpha$

$$
\therefore \alpha \in \gamma = \alpha < \gamma
$$

\fim

\end{document}

