\documentclass[12pt]{extarticle}
\usepackage[brazil]{babel}
\usepackage{graphicx, hyperref}
\usepackage{framed}
\usepackage[normalem]{ulem}
\usepackage{amsmath}
\usepackage{amsthm}
\usepackage{amssymb}
\usepackage{amsfonts}
\usepackage{enumitem}
\usepackage{enumerate}
\usepackage[utf8]{inputenc}
\usepackage[top=1 in,bottom=1in, left=1 in, right=1 in]{geometry}
\usepackage{graphicx}
\usepackage[skip=2pt]{caption}
\usepackage{blindtext}
\usepackage[fixlanguage]{babelbib}
\usepackage{chngcntr}
\usepackage{calrsfs}
\captionsetup{font={small}} 
\graphicspath{{Figuras/}}
\usepackage{appendix}
\newcommand{\asp}[1]{``#1"}
\usepackage[colorinlistoftodos]{todonotes}
\newcommand{\nota}[1]{\todo[color=black!10, bordercolor=white!100, linecolor = black!20, size =\scriptsize]{#1} }
\definecolor{mypink1}{rgb}{0.858, 0.188, 0.478}
\definecolor{mypink2}{RGB}{219, 48, 122}
\definecolor{mypink3}{cmyk}{0, 0.7808, 0.4429, 0.1412}
\definecolor{mygray}{gray}{0.6}
\newcommand{\arb}{um elemento arbitrário }
\newenvironment{resposta}{ \color{mygray}}{}
\newcommand{\true}{\textcolor{red}{\textbf{\textit{V}}}}
\newcommand{\false}{\textcolor{red}{\textbf{\textit{F}}}}
\newcommand{\keys}[1]{\{#1\}}
\newcommand{\subscript}[2]{$#1 _ #2$}
\newcommand{\definition}[3][x]{\{#1|#1 \in \mathbb{#2},#3\}}
\newcommand{\natura}{\mathbb{N}}
\newcommand{\integer}{\mathbb{Z}}
\newcommand{\real}{\mathbb{R}}
\newenvironment{pif}[3][1]{
Tomando $n = #1$ então $#2$ e $#3$

Agora vamos assumir que a expressão é valida para $#1 > n \leq k$ e vamos provar que funciona para $n=k+1$.
}{}
\newcommand{\arranjo}[2]{A_{#1,#2}=\frac{#1!}{(#1-#2)!}}
\newcommand{\arranjoform}[2]{\frac{#1!}{(#1-#2)!}}
\newcommand{\conbination}[2]{C_{#1,#2}=\frac{#1!}{#2!(#1-#2)!}}
\newcommand{\conbinationform}[2]{\frac{#1!}{#2!(#1-#2)!}}
\newcommand{\rconbination}[2]{CR_{#1,#2}=\frac{(#1+#2-1)!}{#2!(#1-1)!}}
\newcommand{\clinear}[1]{\foreach \i in {1,...,#1}{x_\i+}}
\newcommand{\fatorial}[2]{\foreach \i in {#1,...,#2}{\i.}}
\newcommand{\soma}[2]{\sum_{n=#1}^{\infty}#2}
\newcommand{\texto}{\color{mygray}\blindtext\color{black}}
\newcommand{\fim}{\begin{flushright}

   \emph{q.e.d}
\end{flushright}}
\counterwithin*{equation}{section}
\counterwithin*{equation}{subsection}
%--------------%--------------%----------------------%-----------%--------------

\title{Teoria Axiomática dos Conjuntos}
\author{Beatriz de Faria, 11201810015}
\date{Março, 2021}

\begin{document}

\maketitle

\section{Exercício 3.3.}

O conjunto $\bigcap A$ é definido por:

$$
x \in \bigcap A \Leftrightarrow \forall a \in A (x \in a)
$$

Queremos provar que:\\

\textbf{(i) $\emptyset \in x$}

De fato, como $ a \in A \Rightarrow$ a é indutivo. Então, pela definição de conjunto indutivo, $\forall a \in A (\emptyset \in a)$,isto é, $\emptyset \in x$. \\

\textbf{(ii) $z \in x \Rightarrow s(z) \in x$}

Fixe um $z \in x$ qualquer, como $z \in x \Rightarrow \forall a \in A (z \in a).$ Só que, $ a \in A \Rightarrow$ a é indutivo. Ou seja, $\forall a \in A (s(z) \in a)$. Portanto $s(z) \in x$.

\fim

\section{Exercício 3.14.}

Por definição:

$$
m \leq n \Leftrightarrow (m \in n \lor m = n)
$$

Disso, temos que:

$$
m < n \Leftrightarrow (m \leq n \land m \neq n)
$$

é equivalente a dizer:

$$
m < n \Leftrightarrow ((m \in n \lor m = n) \land m \neq n)
$$

Note que não pode ocorrer m = n. Portanto, $m \in n$

\section{Exercício 3.17.}

Fixe $n \in \omega$ qualquer

$(\Rightarrow)$ 

Como $\omega$ é um conjunto indutivo, $\forall x \in \omega, s(x) \in \omega$. Suponha, por absurdo, que $\nexists k : s(k) = n$, além disso, $\forall a$ tal que a é um conjunto indutivo $\omega \subseteq a$. Seja a o conjunto construído da seguinte forma: $\forall y (y \in a \Leftrightarrow y = \emptyset \lor$ y é o sucessor de algum elemento de a). Note que existe um conjunto indutivo que satisfaz as hipóteses colocadas. 

Como $\omega \subseteq a$, então $\forall y \in \omega (y \in a)$. Então, como n não pode ser o sucessor de um elemento k e $\forall y \in \omega (y = \emptyset \lor y = s(k))$ para algum k em $\omega$. Disto temos que $n = \emptyset$ o que contradiz nossa hipótese.

$(\Leftarrow)$ 

Essencialmente a mesma coisa feita anteriormente só que sem a parte do absurdo.

\fim

\section{Exercício 3.18.}

\textbf{(a)}

Fixe um $n \in \omega$ qualquer.

Como $\omega$ é indutivo, $s(n) \in \omega$. Pela nossa definição de sucessor:

$$
s(n) = n \cup \{n\} = \{n ,\{n\}\}
$$

O axioma das partes nos dá que:

$$
\therefore n \in s(n) \land s(n) \in \omega \Rightarrow n \subseteq \omega
$$
\\
\textbf{(b)}
\\
\textbf{(c)}

\section{Exercício 4.17.}

\section{Exercício 4.21.}

\section{Exercício 5.5.}

(A ordem das demonstrações nesse exercicio vai ficar meio esquisita, mas, vamos lá

\textbf{(i) $\Rightarrow$ (iii)}

Nossa hipótese é que:

$$
(x \leq y \land x \neq y)
$$

Pelo modo como definimos $x \leq y$:

$$
((x \in y \lor x = y) \land x \neq y)
$$

$$
((x \in y \land x \neq y) \lor (x = y \land x \neq y) 
$$

A segunda condição no nosso \textbf{\asp{ou}} não pode ocorrer, portanto, ocorre que:

$$
(x \in y \land x \neq y)
$$
 
 Em particular:
 
$$
x \in y
$$
\\
\textbf{(i) $\Rightarrow$ (ii)}

Queremos provar que:

$$
\neg (y \in x \lor y = x)
$$

que é a mesma coisa que:

$$
(y \notin x \land y \neq x)
$$

Primeiramente, vamos provar que $y \neq x$. Por hipótese, temos: 

$$
x \leq y \land x \neq y
$$

Portanto, $y \neq x$ decorre diretamente da nossa hipótese.

Agora, vamos provar que $y \notin x$. Já sabemos que $(i) \Rightarrow (iii)$, então temos que $x \in y$. Suponha por absurdo, que $y \in x$. Isto é:

$$
x \in y \land y \in x
$$



\textbf{() $\Rightarrow$ ()}
\\
\textbf{() $\Rightarrow$ ()}
\\
\section{Exercício 5.13.}



\end{document}
